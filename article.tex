% https://en.wikibooks.org/wiki/LaTeX/Document_Structure#Document_classes



% \documentclass{article}

\documentclass[11pt,twocolumn,a4paper]{article}



\usepackage{blindtext}



\begin{document}



\title{DS200 Module 4: Assignment}

\author{Nashez Zubair\\ nashezzubair@iisc.ac.in}



\maketitle



\begin{abstract}

Your abstract goes here...

...

\end{abstract}



\section{Introduction}



\section{Review of Paper 1}

Starting a new section here.

This is a review of~\cite{perera:2016}.



\subsection{Strengths}
Proposes a model that allows non IT experts to configure sensorsand data processing mechanisms in an IoT middleware(the CASCOM model). The approach is scalable, autonomous, utility based,dynamic and easy to use. Also performs additional context discovery and provides suggestions.
\subsection{Weakness}
The Question-Answer based model might not be generalizable without lot of efforts. Also not much to do here for an IT expert.
\subsection{Summary}
Not considering any privacy aspects into the model. CASCOM significantly useful for non IT experts and identifying secondary context information.




\section{Review of Paper 2}

This is a review of~\cite{Hartig:2009}.



\subsection{Strengths}
Proposes a data provenance model. Information Quality is seen as an aggregated value of multiple criteria. A graph based approach so easily visualisable.
\subsection{Weakness}
Tailors a data provenance model specifically for the use case.Does not depend on the granularity.
\subsection{Summary}
An approach for using provenance information about the data on the Web to assess their quality and trustworthiness.


\section{Conclusions}



\bibliographystyle{IEEEtran}

\bibliography{article}



\end{document}